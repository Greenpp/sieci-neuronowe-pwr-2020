\documentclass{article}
\usepackage[final]{nips_2017}
\usepackage{polski}
\usepackage[utf8]{inputenc}    % allow utf-8 input
\usepackage[T1]{fontenc}       % use 8-bit T1 fonts
\usepackage{hyperref}          % hyperlinks
\usepackage{url}               % simple URL typesetting
\usepackage{booktabs}          % professional-quality tables
\usepackage{amsfonts}          % blackboard math symbols
\usepackage{nicefrac}          % compact symbols for 1/2, etc.
\usepackage{microtype}         % microtypography
\usepackage[section]{placeins} % figures kept in sections
\usepackage{graphicx}          % images
\graphicspath{ {./img/} }
\usepackage{multirow}
\usepackage{float}             % figures in place
\usepackage{caption}		   % smaller margin after figure

\renewcommand{\figurename}{Wykres}
\setlength{\belowcaptionskip}{-20pt}

\title{  Sieć konwolucyjna\\Sieci Neuronowe 2020 }

\author{
  Jakub Ciszek \\
  238035\\
}

\begin{document}

\maketitle

\newpage
\tableofcontents
\newpage

Cały kod wykorzystany w zadaniu znajduje się pod adresem: \url{https://github.com/Greenpp/sieci-neuronowe-pwr-2020}

\section{Opis badań}
\subsection{Plan eksperymentów}

Wszystkie eksperymenty zostały przeprowadzone 10 razy. Losowość przy inicjalizacji wag oraz generacji danych nie została narzucona żadnym ziarnem. Podczas badań przyjęto górną granicę 5 epok, po przekroczeniu której, uczenie zostawało przerywane. Ze względu na charakter zadania (klasyfikacja) na ostatniej warstwie użyto funkcji Softmax, a za funkcję straty przyjęto Entropię krzyżową. Użyta sieć składała się z warstwy konwolucyjnej, max pool, oraz w pełni połączonej ze 128 neuronami. Wagi były inicjalizowane metodą He, a uczenie przebiegało przy pomocy metody Adam. Jako funkcję aktywacji przyjęto ReLU.
TODO model MLP
Z powodów wydajnościowych testowanie modelu przeprowadzano co każde 32 paczki, z których każda składała się z 32 przykładów.\\
Zgodnie z instrukcją zostały przeprowadzone następujące badania:
\begin{itemize}
	\item Wpływ wielkości filtra na przebieg procesu uczenia
	\item Porównanie z MLP    
\end{itemize}
Podczas wizualizacji funkcji straty pominięto pierwsze 10 pomiarów dla lepszej czytelności.

\subsection{Charakterystyka zbiorów danych}

Danymi użytymi w zadaniu jest zbiór ręcznie pisanych cyfr \(0-9\) - MNIST. Na zbiór składa się 70,000 obrazów wielkości 28x28 pikseli. Na wyjściu znajduje się 10 klasom na wyjściu. Użyta w zadaniu wersja została podzielona na 3 zbiory:
\begin{itemize}
	\item Uczący - 50,000 przykładów.
	\item Walidujący - 10,000 przykładów.
	\item Testowy - 10,000 przykładów.
\end{itemize}
W trakcie eksperymentów wykorzystano jedynie zbiory uczący i testowy.

\newpage
\section{Eksperymenty}

\subsection{Wpływ wielkości filtra na przebieg procesu uczenia}
\subsubsection*{Założenia}

Zmienną w tym eksperymencie była wielkość filtra, przyjmowała wartości ze zbioru \(\{$3, 5, 7$\}\)
\subsubsection*{Przebieg}

Podczas eksperymentu model został zainicjalizowany 10 razy dla każdej z badanych wartości oraz wyuczony, uzyskane wyniki zostały zapisane w postaci pliku .plk do dalszej analizy.

\subsubsection*{Wyniki}
\begin{figure}[H]
	\centering
	\caption{Dokładność modelu w zależności od wielkości filtra}
	\includegraphics[width=\textwidth]{kernel_acc.png}
	\label{fig:res11}
\end{figure}
\begin{figure}[H]
	\centering
	\caption{Dokładność modelu w końcowym etapie uczenia w zależności od wielkości filtra}
	\includegraphics[width=\textwidth]{kernel_acc_zoom.png}
	\label{fig:res12}
\end{figure}
\begin{figure}[H]
	\centering
	\caption{Zachowanie funkcji błędu dla filtra wielkości 3}
	\includegraphics[width=\textwidth]{kernel_err_3.png}
	\label{fig:res13}
\end{figure}
\begin{figure}[H]
	\centering
	\caption{Zachowanie funkcji błędu dla filtra wielkości 5}
	\includegraphics[width=\textwidth]{kernel_err_5.png}
	\label{fig:res14}
\end{figure}
\begin{figure}[H]
	\centering
	\caption{Zachowanie funkcji błędu dla filtra wielkości 7}
	\includegraphics[width=\textwidth]{kernel_err_7.png}
	\label{fig:res15}
\end{figure}

\begin{table}[H]
	\caption{Średnia maksymalna dokładność w zależności od wielkości filtra}
	\label{tabela-res-11}
	\centering
	\begin{tabular}{rrr}
		\toprule
		Wielkość filtra & Dokładność [\%] \\
		\midrule
		3                 & \textbf{98.25}     \\
		5                 & 98.16              \\
		7                 & 97.88              \\
		\bottomrule
	\end{tabular}
\end{table}

\subsubsection*{Wnioski}

TODO

\newpage
\subsection{Porównanie z MLP}
\subsubsection*{Założenia}
\begin{table}[H]
	\caption{Stałe dla eksperymentu 2}
	\label{tabela-const-2}
	\centering
	\begin{tabular}{lr}
		\toprule
		Parametr          & Wartość \\
		\midrule
		Wielkość filtra & 3         \\
		\bottomrule
	\end{tabular}
\end{table}

\subsubsection*{Przebieg}

Podczas eksperymentu model został zainicjalizowany 10 razy dla każdej z badanych wartości oraz wyuczony, uzyskane wyniki zostały zapisane w postaci pliku .plk do dalszej analizy.

\subsubsection*{Wyniki}
\begin{figure}[H]
	\centering
	\caption{Porównanie dokładności modeli}
	\includegraphics[width=\textwidth]{con_mlp_acc.png}
	\label{fig:res21}
\end{figure}
\begin{figure}[H]
	\centering
	\caption{Porównanie dokładności modeli w końcowym etapie uczenia}
	\includegraphics[width=\textwidth]{con_mlp_acc_zoom.png}
	\label{fig:res22}
\end{figure}
\begin{figure}[H]
	\centering
	\caption{Zachowanie funkcji błędu dla modelu MLP}
	\includegraphics[width=\textwidth]{mlp_err.png}
	\label{fig:res23}
\end{figure}
\begin{figure}[H]
	\centering
	\caption{Zachowanie funkcji błędu dla modelu konwolucyjnego}
	\includegraphics[width=\textwidth]{con_err.png}
	\label{fig:res24}
\end{figure}

\begin{table}[H]
	\caption{Średnia maksymalna dokładność w zależności od modelu}
	\label{tabela-res-21}
	\centering
	\begin{tabular}{rrr}
		\toprule
		Przykłady   & Dokładność [\%] \\
		\midrule
		MLP          & 97.70              \\
		Konwolucyjna & \textbf{98.25}     \\
		\bottomrule
	\end{tabular}
\end{table}

\subsubsection*{Wnioski}

TODO

\newpage
\section{Wnioski}

\begin{itemize}
	\item TODO
\end{itemize}

\end{document}