\documentclass{article}
\usepackage[final]{nips_2017}
\usepackage{polski}
\usepackage[utf8]{inputenc}    % allow utf-8 input
\usepackage[T1]{fontenc}       % use 8-bit T1 fonts
\usepackage{hyperref}          % hyperlinks
\usepackage{url}               % simple URL typesetting
\usepackage{booktabs}          % professional-quality tables
\usepackage{amsfonts}          % blackboard math symbols
\usepackage{nicefrac}          % compact symbols for 1/2, etc.
\usepackage{microtype}         % microtypography
\usepackage[section]{placeins} % figures kept in sections
\usepackage{graphicx}          % images
\graphicspath{ {./img/} }
\usepackage{multirow}
\usepackage{float}             % figures in place
\usepackage{caption}		   % smaller margin after figure

\renewcommand{\figurename}{Wykres}
\setlength{\belowcaptionskip}{-20pt}

\title{  Optymalizacja uczenia\\Sieci Neuronowe 2020 }

\author{
  Jakub Ciszek \\
  238035\\
}

\begin{document}

\maketitle

\newpage
\tableofcontents
\newpage

Cały kod wykorzystany w zadaniu znajduje się pod adresem: \url{https://github.com/Greenpp/sieci-neuronowe-pwr-2020}

\section{Opis badań}
\subsection{Plan eksperymentów}

Wszystkie eksperymenty zostały przeprowadzone 10 razy. Losowość przy inicjalizacji wag oraz generacji danych nie została narzucona żadnym ziarnem. Podczas badań przyjęto górną granicę 5 epok, po przekroczeniu której, uczenie zostawało przerywane. Ze względu na charakter zadania (klasyfikacja) na ostatniej warstwie użyto funkcji Softmax, a za funkcję straty przyjęto Entropię krzyżową. Warstwa ukryta składała się z 512 neuronów, a początkowy współczynnik uczenia wynosił 0.01.
Z powodów wydajnościowych testowanie modelu przeprowadzano co każde 32 paczki, z których każda składała się z 32 przykładów.\\
Zgodnie z instrukcją zostały przeprowadzone następujące badania:
\begin{itemize}
	\item Wpływ optymalizatorów na przebieg procesu uczenia
	\item Wpływ inicjalizacji wag na przebieg procesu uczenia 
\end{itemize}
Podczas wizualizacji funkcji straty pominięto pierwsze 10 pomiarów dla lepszej czytelności.

\subsection{Charakterystyka zbiorów danych}

Danymi użytymi w zadaniu jest zbiór ręcznie pisanych cyfr \(0-9\) - MNIST. Na zbiór składa się 70,000 obrazów wielkości 28x28 pikseli, co po przekształceniu odpowiadało 784 elementowemu wektorowi wejściowemu i 10 klasom na wyjściu. Użyta w zadaniu wersja została podzielona na 3 zbiory:
\begin{itemize}
	\item Uczący - 50,000 przykładów.
	\item Walidujący - 10,000 przykładów.
	\item Testowy - 10,000 przykładów.
\end{itemize}
W trakcie eksperymentów wykorzystano jedynie zbiory uczący i testowy.

\newpage
\section{Eksperymenty}

\subsection{Wpływ optymalizatorów na przebieg procesu uczenia}
\subsubsection*{Założenia}
\begin{table}[H]
	\caption{Stałe dla eksperymentu 1}
	\label{tabela-const-1}
	\centering
	\begin{tabular}{lr}
		\toprule
		Parametr          & Wartość         \\
		\midrule
		Inicjalizacja wag & \($-0.1 -- 0.1$\) \\
		\bottomrule
	\end{tabular}
\end{table}

Zmienną w tym eksperymencie był użyty optymalizator uczenia. Użyto metod ze zbioru \(\{$SGD, Momentum, Nesterov, AdaGrad, AdaDelta, Adam$\}\)
\subsubsection*{Przebieg}

Podczas eksperymentu model został zainicjalizowany 10 razy dla każdej z badanych wartości oraz wyuczony, uzyskane wyniki zostały zapisane w postaci pliku .plk do dalszej analizy. Badania wykonano dla funkcji aktywacji Sigmoid oraz ReLU.

\subsubsection*{Wyniki}
\begin{figure}[H]
	\centering
	\caption{Dokładność modelu w zależności od użytego optymalizatora dla funkcji ReLU}
	\includegraphics[width=\textwidth]{opt_relu_acc.png}
	\label{fig:res101}
\end{figure}
\begin{figure}[H]
	\centering
	\caption{Dokładność modelu w końcowym etapie uczenia w zależności od użytego optymalizatora dla funkcji ReLU}
	\includegraphics[width=\textwidth]{opt_relu_acc_zoom.png}
	\label{fig:res102}
\end{figure}
\begin{figure}[H]
	\centering
	\caption{Zachowanie funkcji błędu dla funkcji ReLU i optymalizatora Momentum}
	\includegraphics[width=\textwidth]{relu_mom_err.png}
	\label{fig:res103}
\end{figure}
\begin{figure}[H]
	\centering
	\caption{Zachowanie funkcji błędu dla funkcji ReLU i optymalizatora Nesterov}
	\includegraphics[width=\textwidth]{relu_nes_err.png}
	\label{fig:res104}
\end{figure}
\begin{figure}[H]
	\centering
	\caption{Zachowanie funkcji błędu dla funkcji ReLU i optymalizatora AdaGrad}
	\includegraphics[width=\textwidth]{relu_grad_err.png}
	\label{fig:res105}
\end{figure}
\begin{figure}[H]
	\centering
	\caption{Zachowanie funkcji błędu dla funkcji ReLU i optymalizatora AdaDelta}
	\includegraphics[width=\textwidth]{relu_delta_err.png}
	\label{fig:res106}
\end{figure}
\begin{figure}[H]
	\centering
	\caption{Zachowanie funkcji błędu dla funkcji ReLU i optymalizatora Adam}
	\includegraphics[width=\textwidth]{relu_adam_err.png}
	\label{fig:res107}
\end{figure}
\begin{figure}[H]
	\centering
	\caption{Zachowanie funkcji błędu dla funkcji ReLU i optymalizatora SGD}
	\includegraphics[width=\textwidth]{relu_sgd_err.png}
	\label{fig:res108}
\end{figure}
\begin{figure}[H]
	\centering
	\caption{Zachowanie funkcji błędu dla funkcji ReLU z listy 2}
	\includegraphics[width=\textwidth]{relu_2_err.png}
	\label{fig:res109}
\end{figure}
\begin{figure}[H]
	\centering
	\caption{Dokładność modelu w zależności od użytego optymalizatora dla funkcji Sigmoid}
	\includegraphics[width=\textwidth]{opt_sig_acc.png}
	\label{fig:res110}
\end{figure}
\begin{figure}[H]
	\centering
	\caption{Dokładność modelu w końcowym etapie uczenia w zależności od użytego optymalizatora dla funkcji Sigmoid}
	\includegraphics[width=\textwidth]{opt_sig_acc_zoom.png}
	\label{fig:res111}
\end{figure}
\begin{figure}[H]
	\centering
	\caption{Zachowanie funkcji błędu dla funkcji ReLU i optymalizatora Momentum}
	\includegraphics[width=\textwidth]{sig_mom_err.png}
	\label{fig:res112}
\end{figure}
\begin{figure}[H]
	\centering
	\caption{Zachowanie funkcji błędu dla funkcji ReLU i optymalizatora Nesterov}
	\includegraphics[width=\textwidth]{sig_nes_err.png}
	\label{fig:res113}
\end{figure}
\begin{figure}[H]
	\centering
	\caption{Zachowanie funkcji błędu dla funkcji ReLU i optymalizatora AdaGrad}
	\includegraphics[width=\textwidth]{sig_grad_err.png}
	\label{fig:res114}
\end{figure}
\begin{figure}[H]
	\centering
	\caption{Zachowanie funkcji błędu dla funkcji ReLU i optymalizatora AdaDelta}
	\includegraphics[width=\textwidth]{sig_delta_err.png}
	\label{fig:res115}
\end{figure}
\begin{figure}[H]
	\centering
	\caption{Zachowanie funkcji błędu dla funkcji ReLU i optymalizatora Adam}
	\includegraphics[width=\textwidth]{sig_adam_err.png}
	\label{fig:res116}
\end{figure}
\begin{figure}[H]
	\centering
	\caption{Zachowanie funkcji błędu dla funkcji ReLU i optymalizatora SGD}
	\includegraphics[width=\textwidth]{sig_sgd_err.png}
	\label{fig:res117}
\end{figure}
\begin{figure}[H]
	\centering
	\caption{Zachowanie funkcji błędu dla funkcji Sigmoid z listy 2}
	\includegraphics[width=\textwidth]{sig_2_err.png}
	\label{fig:res118}
\end{figure}

\begin{table}[H]
	\caption{Średnia maksymalna dokładność w zależności od użytego optymalizatora}
	\label{tabela-res-11}
	\centering
	\begin{tabular}{rrr}
		\toprule
		\multirow{2}{*}{Optymalizator} & \multicolumn{2}{c}{Dokładność [\%]} \\
		         & ReLU           & Sigmoid        \\
		\midrule
		Momentum & 93.92          & 89.91          \\
		Nesterov & 93.91          & 89.81          \\
		AdaGrad  & \textbf{97.72} & 94.62          \\
		AdaDelta & 95.28          & 92.21          \\
		Adam     & 93.86          & \textbf{97.15} \\
		SGD      & 93.86          & 89.89          \\
		Lista 2  & 97.70          & 93.29          \\
		\bottomrule
	\end{tabular}
\end{table}

\subsubsection*{Wnioski}

TODO

\newpage
\subsection{Wpływ inicjalizacji wag na przebieg procesu uczenia}
\subsubsection*{Założenia}
\begin{table}[H]
	\caption{Stałe dla eksperymentu 2}
	\label{tabela-const-2}
	\centering
	\begin{tabular}{lr}
		\toprule
		Parametr      & Wartość \\
		\midrule
		Optymalizator & SGD       \\
		\bottomrule
	\end{tabular}
\end{table}

Zmienną w tym eksperymencie był sposób inicjalizacji wag. Użyto metod ze zbioru \(\{$Zakres, Xavier, He$\}\)
\subsubsection*{Przebieg}

Podczas eksperymentu model został zainicjalizowany 10 razy dla każdej z badanych wartości oraz wyuczony, uzyskane wyniki zostały zapisane w postaci pliku .plk do dalszej analizy. Badania wykonano dla funkcji aktywacji Sigmoid oraz ReLU.

\subsubsection*{Wyniki}
\begin{figure}[H]
	\centering
	\caption{Dokładność modelu w zależności od sposóbu inicjalizacji wag dla funkcji ReLU}
	\includegraphics[width=\textwidth]{relu_init_acc.png}
	\label{fig:res201}
\end{figure}
\begin{figure}[H]
	\centering
	\caption{Dokładność modelu w końcowym etapie uczenia w zależności od sposóbu inicjalizacji wag dla funkcji ReLU}
	\includegraphics[width=\textwidth]{relu_init_acc_zoom.png}
	\label{fig:res202}
\end{figure}
\begin{figure}[H]
	\centering
	\caption{Zachowanie funkcji błędu dla funkcji ReLU i inicjalizacji z zakresu}
	\includegraphics[width=\textwidth]{relu_init_zak.png}
	\label{fig:res203}
\end{figure}
\begin{figure}[H]
	\centering
	\caption{Zachowanie funkcji błędu dla funkcji ReLU i inicjalizacji Xaviera}
	\includegraphics[width=\textwidth]{relu_init_xav.png}
	\label{fig:res204}
\end{figure}
\begin{figure}[H]
	\centering
	\caption{Zachowanie funkcji błędu dla funkcji ReLU i inicjalizacji He}
	\includegraphics[width=\textwidth]{relu_init_he.png}
	\label{fig:res205}
\end{figure}
\begin{figure}[H]
	\centering
	\caption{Dokładność modelu w zależności od sposóbu inicjalizacji wag dla funkcji Sigmoid}
	\includegraphics[width=\textwidth]{sig_init_acc.png}
	\label{fig:res206}
\end{figure}
\begin{figure}[H]
	\centering
	\caption{Dokładność modelu w końcowym etapie uczenia w zależności od sposóbu inicjalizacji wag dla funkcji Sigmoid}
	\includegraphics[width=\textwidth]{sig_init_acc_zoom.png}
	\label{fig:res207}
\end{figure}
\begin{figure}[H]
	\centering
	\caption{Zachowanie funkcji błędu dla funkcji ReLU i inicjalizacji z zakresu}
	\includegraphics[width=\textwidth]{sig_init_zak.png}
	\label{fig:res208}
\end{figure}
\begin{figure}[H]
	\centering
	\caption{Zachowanie funkcji błędu dla funkcji ReLU i inicjalizacji Xaviera}
	\includegraphics[width=\textwidth]{sig_init_xav.png}
	\label{fig:res209}
\end{figure}
\begin{figure}[H]
	\centering
	\caption{Zachowanie funkcji błędu dla funkcji ReLU i inicjalizacji He}
	\includegraphics[width=\textwidth]{sig_init_he.png}
	\label{fig:res210}
\end{figure}

\begin{table}[H]
	\caption{Średnia maksymalna dokładność w zależności od sposóbu inicjalizacji wag}
	\label{tabela-res-21}
	\centering
	\begin{tabular}{rrr}
		\toprule
		\multirow{2}{*}{Inicjalizacja} & \multicolumn{2}{c}{Dokładność [\%]} \\
		       & ReLU           & Sigmoid        \\
		\midrule
		Zakres & 97.70          & 93.29          \\
		Xavier & 97.71          & \textbf{97.18} \\
		He     & \textbf{97.73} & 97.16          \\
		\bottomrule
	\end{tabular}
\end{table}

\subsubsection*{Wnioski}

TODO


\newpage
\section{Wnioski}

\begin{itemize}
	\item % TODO
\end{itemize}

\end{document}